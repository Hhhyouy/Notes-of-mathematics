\section{前言}
本来洋洋洒洒落笔千言,最后删减到现在的三言两语,我还是决定把最纯粹的内容奉献给大家。我写这份笔记,一方面是个人学习的总结,便于不断地复习回顾;另一方面也是希望能够给那些在学习数学上遇到困难的同道中人一些微不足道的帮助;最后,比较详尽的公式让那些有教学、科研需要的朋友们可以到源文件内直接复制粘贴,大大减轻了他们的工作量,也能更加美观。本笔记错误、疏漏在所难免,希望大家谅解。

这份笔记是由\LaTeX{}编写的,我认为\LaTeX{}有更自由的方式来实现我偏好的设计,同时其对于数学公式的排版也是诸如Word、Indesign等无法媲美的。在笔记的体例上,大致是正文和特殊环境的组合。特殊环境中,定义被环境左侧的视觉引导线包裹,证明结束的标志是实心的小正方形$\blacksquare $,定理和例题的分别是灯泡和图钉。这样,正文和特殊环境总是能够区分开来。同时,我定义了一系列的命令如:\jie \zheng \txe{注:} 等等来方便录入。笔记的内容参考了本科数学教材、考研教辅资料和真题等等。

另外,本笔记内容(包括.tex源文件)全部开源,并遵循CC协议。

\hfill 可达可达

\hfill 2023年1月2日